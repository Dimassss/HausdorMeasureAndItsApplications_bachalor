\chapter*{Wstęp}

\addcontentsline{toc}{chapter}{Wstęp}

W tej prace będzie opisana miara Hausdorffa i jej zastosowanie do mierzania zbiorów na podrozmaitościach w przestrzeni $\mathbb{R}^n$.

W tym celu w pierwszym rozdziale wprowadzimy konstrukcę Caratheodory'ego. Potem zbudujemy na niej miary Hausdorffa i Lebesgua oraz zbadamy własności tych miar, które będą wykorzystywane w dalszych dowodach.

W następnym rozdziale wprowadzimy definicę miary Lebesgue'a na podrozmaitościach. Jednocześnie zdefinujemy Jacobian i udowodnimy równoważność tej definicji do definicji Federa \citep{Federer}. Potem pokażemy ważne własności tej miary i jej $\sigma$-algebry. W końcu drugiegu rozdiału udowodnimy twierdzenie, które pokazauje jak liczyć miarę Lebesgue'a na podrozmaitościach.

W trzecim rozdziale na wstępie udowodnimy ważne fakty przygotowawcze dotyczące funckji klasy $C^1$ 
dla dowodu głównego twierdzenia tej pracy proseminaryjnej. Potem udowodnimy twierdzenie dotyczące 
tego jak liczyć miarę Hausdorffa obrazu funckji klasy $C^1$. Na podstawie tego twierdzenia oraz 
twierdzenia z drugiego rozdziału o tym jak liczyć miarę Lebesgue'a na podrozmaitościach dowiemy się, 
że miara Lebesgue'a na podrozmaitościach i miara Hausdorffa to jest to samo, co pozwala nam wykorzystywać 
wszelkie własności miary Hausdorffa dla liczenia miar zbiorów i całek na podrozmaitościach w $\mathbb{R}^n$ .
