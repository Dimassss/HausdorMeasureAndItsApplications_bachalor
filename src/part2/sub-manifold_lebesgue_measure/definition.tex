\begin{defi}[Zbiory mierzalne na podrozmaitościach]
    Niech $M \in \mathcal{M}_n^1(\mathbb{R}^m)$ , to znaczy $M$ jest $n$-wymiarową podrozmaitością w $\mathbb{R}^m$ klasy $C^1$.
    \begin{itemize}
        \item Jeżeli $n=0$, to przez miarę Lebesgue'a na $M$ rozumiemy miarę liczącą. Kładziemy $L_M = \mathcal{P}(\mathbb{R}^m)$
        \item W przypadku gdy $n=m$ podrozmaitość $M$ jest zbiorem otwartym w $\mathbb{R}^n$. Wtedy przyjmujemy $\mathcal{L}^M := \mathcal{L}^m$
        \item Przypadek $1 \leq n < m$. Zdefinujemy $L_M$ jako rodzinę wszystkich $A \subset M$ takich, że dla dowolnej lokalnej parametryzacji $p : P \rightarrow U$ mamy $p^{-1}(A) \in L_n$. 
    \end{itemize}
\end{defi}