Dalej pokażamy, że istnieje miara $\mathcal{L}^M$ na tych podrozmaitościach, która, jak w następnych rozdziałach się dowiemy, jest równa mierze Hausdorffa.
Dla przypadku $n=0$, niech $\mathcal{L}^M$ będzie miarą liczącą oraz dla przypadku $n=m$ - miarą Lebesgue'a.