\begin{tw}
    $L_M$ jest $\sigma$-algebrą oraz $\mathcal{B}(\mathbb{R}^m) \subset L_M$.
    \newline
    \textbf{Dowód:}\newline
    Dla $n \in \{0, m\}$ to twierdzenie jest proste. Niech $0 < n < m$. Oczywiście $\emptyset \in L_M$. Ustalmy $A \in L_M$. 
    Pokażemy, że $A^C \in L_M$ . Ustalamy dowolną parametryzację $p: P \rightarrow U$ . Mamy $p^{-1}(A^C) = p^{-1}(M \backslash A) = p^{-1}(M) \backslash p^{-1}(A) = P \backslash p^{-1}(A)$ . Poniważ $P$ jest otwarty oraz $p^{-1}(A) \in L_d$ to stąd wynika, że $p^{-1}(A^C) \in L_n$ .
    Teraz pokażemy, że przeliczalna suma $\{A_j\}_{j \in \mathbb{N}} \subset L_M$ jest w $L_M$ co kończy dowód. Ustalamy dowolne $p: P \rightarrow U$. $$ 
    p^{-1}(\bigcup_{i \in \mathbb{N}} A_i ) = \bigcup_{i \in \mathbb{N}} p^{-1}(A_i) \in L_n
    $$ Zostało pokazać, że zbiory borelowske są w tej algebrze. Wystarczy pokazać, że zbiory otwarte są w $L_M$, ale widać, że tak jest z samej postaci tych parametryzacji. To kończy dowód.  
\end{tw}