\begin{tw}
	$\Omega \in top(\mathbb{R}^n)$, $f: \Omega \rightarrow \mathbb{R}^m$ klasy $C^1$, $1 \leq n \leq m$ . $\forall x \in \Omega: f'(x)$ jest monormifizmem. Niech $L_n \ni A \subset \Omega$ taki, że $f|_A$ - injekcja. Wtedy:
	\begin{enumerate}	
		\item $f(A) \in H_n$,
		\item $\mathcal{H}^n(f(A)) = \int_A J_nfd \mathcal{L}^n$ .
	\end{enumerate}
	
	\textbf{Dowód:}\newline
	Już wiemy, że $f(A) \in H_n$, więc musimy jedynie pokazać tą równość. Ustalmy $\lambda > 1$ i wybieramy rozbicie $\Omega$ oraz ciągi operatorów zgodnie z lematem 3.1.2 $\{(B_v, T_v)\}_{v \in \mathbb{N}}$. Korzystając z własności normy w algebrze zewnętrznej oraz tego, że $J_nf = \|\bigwedge_nf\|$, dostajemy dla $x \in B_v$ oraz $v \in \mathbb{N}$ nierówność 
	$$
		\lambda^{-n}|T_v| \leq (J_nf)(x) \leq \lambda^n|T_v|
	$$
	która wynika bezpośrednio z następującej nierówności z lematu 3.1.2 
	$$
		\lambda^{-1}|T_v(h)| \leq |f'(x)(h)| \leq \lambda|T_v(h)| \quad\quad  \text{dla} \; x \in B_v, h \in \mathbb{R^n}, v \in \mathbb{N}
	$$
	Definujemy $A_v := A \cap B_v$. Wtedy całkując otrzymaną nierówność na $A_v$ dostajemy $$
		\lambda^{-n}|T_v|\mathcal{L}^n(A_v) \leq \int_{A_v}J_nf d \mathcal{L}^n \leq \lambda^n |T_v| \mathcal{L}^n(A_v)
	$$
	W tym samym lemacie 3.1.2 mamy następującą nierówność $$
		\lambda^{-1}|T_v(x_1 - x_2)| \leq |f(x_1) - f(x_2)| \leq \lambda |T_v(x_1, x_2)| \quad \quad \text{dla } x_1,x_2 \in B_v  
	$$
	Skąd już wynika, że $$
		\lambda^{-1}|T_v(x_1) - T_v(x_2)| \leq |f(x_1) - f(x_2)| \leq \lambda|T_v(x_1) - T_v(x_2)| \quad \quad \text{dla } x_1, x_2 \in A_v
	$$
	Rozpatrzymy diagram

	$$
	% https://tikzcd.yichuanshen.de/#N4Igdg9gJgpgziAXAbVABwnAlgFyxMJZABgBpiBdUkANwEMAbAVxiRABUB9GgCgEFuAShABfUuky58hFACZyVWoxZsAZvyGjxIDNjwEi82YvrNWiEAJpaJe6UQAsC6qZUWuvK8LG2pBlACMzkpmbAAW3KKKMFAA5vBEoKoAThAAtkhBIDgQSGQgDHQARjAMAAqS+jIFMKo4NiAp6Ujy2bmI+YUl5ZX2FslYsWH1LsrmHNwAPpzAViINTRmIWTlIAMyjoRYR1j6NqUutq4gbIW4TNNOz3PN7iy3Ux1mu46pXc1EiQA
	\begin{tikzcd}
		T_v(A_v) &    & f(A_v) \arrow[ll, "h_v"] \arrow[rr, "h_v"]                                               &  & T_v(A_v) \\
				&     &                                                                                   &  &          \\
				&     & A_v \arrow[lluu, "T_v|_{A_v}"'] \arrow[rruu, "T_v|_{A_v}"] \arrow[uu, "f|_{A_v}"] &  &         
	\end{tikzcd}
	$$



	dla $h_v = (T_v|_{A_v}) \circ (f|_{A_v})^{-1}$ mamy $Lip(h_v) \leq \lambda$, $Lip h_v^{-1} \leq \lambda$ dla $v \in \mathbb{N}$. Zatem dla $v \in \mathbb{N}$ otrzymujemy $$
		\lambda^{-n} \mathcal{H}^n(T_v(A_v)) \leq \mathcal{H}^n(f(A_v)) \leq \lambda^n \mathcal{H}^n(T_v(A_v))
	$$
	Skąd dostajemy $$
		\lambda^{-2n}\mathcal{H}^n(f(A_v)) \leq \int_{A_v}J_nf d \mathcal{L}^n \leq \lambda^{2n} \mathcal{H}^n(f(A_v))
	$$
	Dalej sumując tą nierówność po $v \in \mathbb{N}$ dostajemy $$
		\lambda^{-2n} \sum_{v=0}^{\infty} \mathcal{H}^n(f(A_v)) \leq \int_{A_v}J_nf d \mathcal{L}^n \leq \lambda^{2n} \sum_{v=0}^{\infty} \mathcal{H}^n(f(A_v))
	$$
	Skąd na podstawie injektywności $f$ dostajemy $$
		\lambda^{-2n}\mathcal{H}^n(f(A)) \leq \int_{A}J_nf d \mathcal{L}^n \leq \lambda^{2n} \mathcal{H}^n(f(A))
	$$
	Teraz widzimy, że jeśli $\mathcal{H}^n(f(A)) = +\infty$ to równość z tezy zachodzi. Jeśli $\mathcal{H}^n(f(A)) < +\infty$ to przechodząc z $\lambda > 1$ do $1$ dostajemy równość $$
		\mathcal{H}^n{f(A)} = \int_A J_nf d\mathcal{L}^n
	$$
\end{tw}