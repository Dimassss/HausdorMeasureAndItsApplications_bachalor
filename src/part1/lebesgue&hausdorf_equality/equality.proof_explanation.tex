Dowód tego twierdzenia polega na wykorzystywaniu nierówności izodiametrycznej \citep[2.10.3]{Federer} 
i twierdzeniu Vitallego \citep[2.8.18]{Federer}.
Spoczątku pokazujemy, że równość jakieś miary $\mu$ na $\mathbb{R}^n$ na kostkach symetrycznych do miary 
Lebesgue'a implikuje, że $\mu$ jest miarą Lebesgue'a, co było zrobione w lemacie \citep[3.35]{Tworzewski}.
Potem biorąc kostku i wykorzystując twierdzenie Vitallego, które mówi że możemy znalieźć takie pokrycie kulami domkniętymi
rozłącznymi o średnice co najwyżej $\epsilon$ tej kostki, że miara Lebesgue'a ich różnicy będzie równa 0. 
To pozwala na pokazanie nierówności w pierwszą stronę: 
$$ 
    \mathcal{H}^m_{\epsilon}(\bigcup_{n=0}^{\infty} B_n) 
    \leq \sum_{n=1}^{\infty} \alpha(m)2^{-m}(diam(B_n))^m 
    = \sum_{n=1}^{\infty} \mathcal{L}^m(B_n) 
    = \mathcal{L}^m(C)
    = 1
$$
gdzie $\{B_n\}_{n \in \mathbb{N}}$ to są te kule oraz $1$ na końcu, to jest objętość tej kostki $C$ (możemy założyć, że jest symetryczna). 
Skąd mamy nierówność $\mathcal{H}^m(C) \leq 1$. Żeby pokazać nierówność w drugą stronę będziemy potrzebowali jedynie tylko
nierówności izodiametrycznej, która mówi że 
$$
    \mathcal{L}^m(A) \leq \alpha(m)2^{-m}diam(A) \quad \quad , A \subset \mathbb{R}^m
$$
Oczywiście, biorąc dowolne pokrycie $\{Y_n\}_{n \in \mathbb{N}}$ kostki $C$ o średnice co najwyżej $\epsilon > 0$ wynika
$$ 
    1 
    \leq \sum_{n=0}^{+\infty} \mathcal{L}^m(Y_n) 
    \leq \sum_{n=0}^{+\infty} \alpha(m)2^{-n}(diam(Y_n))^m 
$$ 
Co z samej konstrukcji miary Hausdorffa daje nam żądaną nierówność $1 \leq \mathcal{H}^m(C)$.