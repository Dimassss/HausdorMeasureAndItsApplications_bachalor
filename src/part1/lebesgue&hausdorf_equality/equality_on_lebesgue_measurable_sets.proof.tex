Na początku zauważmy, że $L_m \subset H_m$. Łatwo widać, że zbiory miary zero w sensie Lebesgue'a są takimi i w sensie Hausdorffa. Bo zaswsze możemy znalieźć pokrycie miary $\epsilon$ takiego zbioru, że jest ono równierz miary co najwyżej $C \epsilon$ dla miary Hausdorffa, gdize $C >> 1$ jest jakaś ustaliona stałą. Zgodnie z powyższymi twierdzeniami i lematami wystaczy pokazać, że równość zachodzi na kostkach domkniętych symetrycznych. Ponieważ te miary są translatywne oraz dla każdego zbioru zachodzi $\mathcal{L}^m(sY) = s^m\mathcal{L}^m(Y)$ oraz  $\mathcal{H}^m(sY) = s^m\mathcal{H}^m(Y)$  wynika, że równość wystarczy pokazać dla kostki $C:= [0,1]^m$ . Czyli pokazać, że $\mathcal{H}^m(C) = s^m\mathcal{L}^m(C) = 1$ . Z twierdzenia Vitallego znajdziemy $\epsilon > 0$ ciąg kul $\{B_n\}$ parami rozłącznych we wnętrzu kostki $C$ o średnicach mniejszych od $\epsilon$ takie, że $$ \mathcal{L}^m(int(C) \backslash \bigcup_{n=0}^{\infty}) = 0 $$ więc $$ \mathcal{H}^m_{\epsilon}(C \backslash \bigcup_{n=0}^{\infty} B_n) = 0 $$ $$ \mathcal{H}^m_{\epsilon}(\bigcup_{n=0}^{\infty} B_n) \leq \sum_{n=1}^{\infty} \alpha(m)2^{-m}(diam(B_n))^m = \sum_{n=1}^{\infty} \mathcal{L}^m(B_n) = 1$$ więc $\mathcal{H}^m_{\epsilon}(C) \leq 1$ dla $\epsilon > 0$ więc stąd otrzymujemy , że $$\mathcal{H}^m(C) \leq 1 $$ Weżmy następnie dla $\epsilon >0 $ ciąg zbiorów $\{Y_n\}$ taki, że $C \subset \bigcup_{n=0}^{\infty} Y_n, diam(Y_n) < \epsilon$ dla $n \in \mathbb{N}$ $$ 1 \leq \sum_{n=0}^{+\infty} \mathcal{L}^m(Y_n) \leq \sum_{n=0}^{+\infty} \alpha(m)2^{-n}(diam(Y_n))^m $$ Zatem $1 \leq \mathcal{H}^m_{\epsilon}(C)$ dla $\epsilon > 0$ skąd wynika, że $\mathcal{H}^m(C) \geq 1$. To wszystko daje nam równość $\mathcal{H}^m(C) = 1$ co kończy dowód.