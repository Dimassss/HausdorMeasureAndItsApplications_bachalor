Wystarczy pokazać, że równość zachodzi dla dowolnych kostek. Ustalimy kostku $I = [a_1, b_1] \times \cdots \times [a_m, b_m]$  . Chcemy znalieżć takie pokrycie kostkami symetrycznymi dla $I$, że $I$ jest sumą tych kostek symetrycznych zwartych o rożłącznych wnętrzach. Niech $a := (a_1, \cdots, a_m)$ wtedy $I-a$ jest taką samą kostką o tej samej objętości, wieć możemy założyć, że $a = 0$ . Bodujemy to pokrycie indukcyjnie. Na pierwszym kroku bierzemy $c_1 := min(b)$. wtedy kostka $[0, c_1]^m \subset I$ .
Teraz definujemy $k_j := \lfloor b_j/c_1 \rfloor$ . Wtedy $K_1 := \prod_{j = 1}^m [0, k_jc] \subset I$   oraz $K_1$ jest sumą skończoną kostek symetrycznych o wnętrzach rozłącznych. $I \backslash K_1$ możemy przedstawić jako suma kostek zwartych o rozłącznych wnętrach. Powtarzamy tą samą procedurę dla tych kostek. Dostajemy ciąg kostek symetrzycznych, które stanowią pokrycie $I$. Brzeg kostki jest miary zero, czyli $I$ możemy przedstawić jako suma wnętrz tych  kostek plus ich brzeg, który jest miary zero, a więc miara $I$ musi być równa mierze sumie tych kostek. Ponieważ $\mu$ jest równa $\mathcal{L}^m$ na tych kostkach symetrzycnych dostajemy, że $\mu(I) = \mathcal{L}^m(I)$ co kończy dowód. 