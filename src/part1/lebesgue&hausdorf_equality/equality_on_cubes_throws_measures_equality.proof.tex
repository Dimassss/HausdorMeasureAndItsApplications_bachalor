Niech $\{I_i\}$ ciąg kostek zwartych który pokrywa $A \in L_m$ . Mamy $$ 
	\mu(A) \leq \sum_{i \in \mathbb{N}} \mu(I_i) = \sum_{i \in \mathbb{N}} vol(I_i)
$$
Zatem $\mu(A) \leq \mathcal{L}^m(A)$ . Zakładając, że $A$ jest ograniczony znajdziemy $I$ przedział zwarty taki, że $A \subset I$. Skoro $$
	\mu(I \backslash A) \leq \mathcal{L}^m(I \backslash A) 
$$ Stąd $\mu(I) - \mu(A) \leq \mathcal{L}^m(I) - \mathcal{L}^m(A)$. Czyli $\mathcal{L}^m(A) \leq \mu(A)$. Zatem dla zbiorów ograniczonych mamy równość $\mathcal{L}^m(A) = \mu(A)$ . Każdy zbiór $A \in L_m$ możemy przedstawić w postaci sumy rozłącznej zbirów mierzlnych ogranicznych skąd wynika teza.