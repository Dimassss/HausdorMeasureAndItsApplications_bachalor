\begin{defi}[Miara Lebesgue'a]
	Niech $\mathcal{F}$ będzie rodzina kostek zwartych w $\mathbb{R}^m$, $m \geq 1$ czyli zbiorów postaci $$
		I = I_1 \times \cdots \times I_m
	$$gdzie $I_j=[a_j, b_j]$ są zwarte przedziały $\mathbb{R}$ dla $j = 1, \dots, m$. Określiamy objętość $I$ jako $$
		vol(I) = (b_1 - a_1) \cdots (b_m - a_m)
	$$
	Dodatkowo przyjmijmy $vol(\emptyset) = 0$\newline
	W ten sposób otrzymujemy funkcję $$
		\zeta: \mathcal{F} \ni I \rightarrow \zeta(I) = vol(I) \in [0, + \infty].
	$$
	Wtedy korzystając z konstrukcji Caratheodory'ego i funkcji $\zeta$ jako funckji tworzącej dostajemy m-wymiarową miarę Lebesgue'a $\mathcal{L}^m$ oraz jej $\sigma$-algebry $L_m$ .
\end{defi}