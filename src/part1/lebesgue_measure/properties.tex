\begin{tw}[Własności miary Lebesgue'a]
    Niech $m \in \mathbb{N} \backslash \{0\}$. Wtedy:
    \begin{enumerate}
    \item $\mathcal{L}^m$ jest miarą zewnętrzną metrzyczną w $\mathbb{R}^m$,
    \item $\mathcal{B}(\mathbb{R}^m) \subset L_m$,
    \item $\mathcal{L}^m$ na $L_m$ jest miarą zupełną ,
    \item Jeśli $I$ kostka, to $\mathcal{L}^m(I) = vol(I)$,
    \item Jeśli $Y$ ograniczony, to $\mathcal{L}^m(Y) < \infty$,
    \item $\mathcal{L}^m$ jest $\sigma$-skończona,
    \item $\forall Y \subset \mathbb{R}^m \exists G \subset \mathbb{R}^m: G$ - typu $G_{\delta} \wedge Y \subset G \wedge \mathcal{L}^m(G) = \mathcal{L}^m(Y)$ ,
    \item $\mathcal{L}^m$ jest miarą zewnętrzną regularną ,
    \item $(Y \subset \mathbb{R}^m \wedge a \in \mathbb{R}^m) \Longrightarrow (\mathcal{L}^m(a + Y) = \mathcal{L}^m(Y))$ ,
    \item $(Y \subset \mathbb{R}^m, s \in [0, +\infty)) \Longrightarrow (\mathcal{L}^m(sY) = s^m\mathcal{L}^m(Y))$.
    \end{enumerate}

    \textbf{Dowód:} \citep[3.41]{Tworzewski}
\end{tw}